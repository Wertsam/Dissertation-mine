\chapter*{Заключение} 
\setcounter{table}{2}                       % Заголовок
\addcontentsline{toc}{chapter}{Заключение}  % Добавляем его в оглавление
В \underline{Главе}~\ref{ch:ch1} проведен обзор современной литературы по тематике квантового распределения ключей, теоретической основы технологии КРК. Описываются различные протоколы квантового распределения ключей: BB84, B92, Measurement-Device-Independent (Недоверенный Приемный Узел), Twin-Field QKD(протокол с использованием 'полей-близницов') и GG02, их технические особенности.
Описываются возможные атаки на техническую реализацию систем квантового распределения ключей, особое внимание уделено атакам на источники лазерного излучения в составе систем КРК.
\newline В \underline{Главе} \ref{ch:ch2} изучается реализация обратной связи для двух источников лазерного излучения в виде оптической инжекции. С помощью этого метода продемонстрирован эффект синхронизации частот лазеров. При помощи этого метода экспериментально реалзиована система квантового распределения ключей на боковых частотах на непрерывных переменных.
В разделе описывается этапы протокола, а также математическая модель детектирования сигналов. В результате продемонстрировано, что при использовании обратной связи в виде оптической инжекции на балансном детекторе наблюдается только одна частота - частота модуляции Алисы. Из этой промежуточной частоты с помощью метода быстрого преобразования Фурье извлечены значения фаз, закодированные Алисой, которые после постобработки преобразуются в значения бит сырого ключа.
\newline В \underline{Главе} \ref{ch:ch3} проводится экспериментальная реализация системы квантового распределения ключей на боковых частотах на непрерывных переменных с применением двух независимых источников лазерного излучения. Для данной системы построена математическая модель детектирования сигналов, демонстирующая все сигналы, формирующиеся в результате взаимодействия двух сигналов на балансном детекторе, из которых выделяется полезный.
Для этой системы также решается проблема компенсации поляризационных искажений из-за прохождения волоконно-оптической линии. Это происходит с помощью контроллера поляризации, который работает по сигналу обратной связи, сформированного на основе результата Быстрого Преобразования Фурье к регестрируемому сигналу и описываются все этапы алгоритма подстройки поляризации. А также показано из какого сигнала возможно извлекать информацию о сыром ключе.
\newline В \underline{Главе} \ref{ch:ch4} описывается новый тип атаки на техническую реализацию систем КРК - атака оптической накачкой. Суть этой атаки заключается в увеличении энергии излучаемых импульсов за счет поглощения более высокочастотного лазерного излучения Евы. 
В ходе работы изучены зависимости Ватт-Амперной характеристики лазера и его дифференциальной квантовой эффективности от мощности накачки Евы. Изучено влияние мощности накачки Евы на выходную среднюю мощность и энергию импульсов Алисы. Определена пороговая мощность на длине волны 1310 нм в $70\mu W$. Проанализирована стойкость промышленной системы КРК к данному типу атаки и произведен расчет минимально необходимой изоляции для защиты от данного типа атаки.
\newline В \underline{Главе} \ref{ch:ch5} изучается атака лазерным 'засевом' на источник лазерного излучения на основе оптической инжекции. В рамках работы изучены зависимости средней мощности, энергии импульсов, длительности импульсов и влияния этой атаки на интерференцию. Показано, что атака 'засевом' увеличивает излучаемую среднюю мощность и энергию импульсов, а также увеличивает среднеквадратическое отклонение этих параметров. 
Показано, что длительность импульсов под действием этой атаки не изменяется. Интерференционная картина под действием атаки ухудшается, что негативно влияет на QBER рабочей системы КРК. Также определена минимальная изоляция, необходимая для защиты источника лазерного излучения на основе оптической инжекции от атаки 'засевом' лазерным излучением.


