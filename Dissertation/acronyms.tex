\chapter*{Список сокращений и условных обозначений} % Заголовок
\addcontentsline{toc}{chapter}{Список сокращений и условных обозначений}  % Добавляем его в оглавление
\noindent
\textbf{КРК} квантовое распределение ключей \\
\textbf{БД} балансный детектор \\
\textbf{ФМ} фазовый модулятор\\
\textbf{СД} светоделитель\\
\textbf{ПОА} перестраиваемый оптический аттенюатор\\ 
\textbf{КК} квантовый канал\\ 
\textbf{ОК} открытый канал\\ 
\textbf{ЛД} лазерный диод\\ 
\textbf{НПУ} недоверенный промежуточный узел\\ 
\textbf{ДП} дискретные переменные\\ 
\textbf{НП} непрерывные переменные\\ 
\textbf{БПФ} быстрое Преобразование фурье\\ 
\textbf{ЛО} локальный осциллятор\\ 
\textbf{ГС} генератор сигналов\\ 
\textbf{ОСЦ} осциллограф\\ 
\textbf{QPSK} quadrature phase-shift keying\\ 
\textbf{QKD} quantum key distribution\\ 
\textbf{ФПВ} функция плотности вероятности\\ 
\textbf{PM} сохранение поляризации\\ 
\textbf{VOA} перестраиваемый оптический аттенюатор\\ 
\textbf{BS} светоделитель\\ 
\textbf{PMCIR} оптический циркулятор с сохранением поляризации\\ 
\textbf{PMBS} светоделитель с сохранением поляризации\\
\textbf{PG} генератор импульсов\\ 
\textbf{OPM} оптический измеритель мощности\\ 
\textbf{OSC} осциллограф\\ 
\textbf{OSA} оптический анализатор спектра\\ 
\textbf{LT} световая ловушка\\ 
\textbf{FM} зеркало Фарадея\\ 
\textbf{DFB} распределенная обратная связь\\ 
\textbf{DWDM} плотное мультиплексирование с разделением по длине волны\\ 
\textbf{ДКЭ} дифференциальная квантовая эффективность\\ 
\textbf{LD} лазерный диод\\ 
\textbf{PS} источник напряжения\\ 
\textbf{OC} оптический циркулятор\\ 
\textbf{PC} контроллер поляризации\\
\textbf{PM} измеритель мощности\\ 
\textbf{PD} фотоприемник\\ 