\chapter*{Введение}                         % Заголовок
\addcontentsline{toc}{chapter}{Введение}
\noindent 
Актуальность темы.\\
Квантовое распределения ключа (КРК) - актуальная технология, развившаяся из теории квантовой информатики, позволяющая распределить симметричную битовую последовательность с помощью квантовых методов у двух и более пользователей для использования этой последовательности в качестве ключа для симметричного шифрования данных и одновременным обнаружением несанкционированного доступа со стороны нелегитимных пользователей. Использование квантовых состояний света при распределении ключа позволяет достичь уровня секретности, недоступного для классических протоколов шифрования. Такие квантовые состояния могут быть представлены в виде одиночных фотонов. Их квантовые свойства не позволяют злоумышленнику скопировать их состояния или счтитать их без изменения и без внесения ошибок. Такие квантовые состояния возможно передавать как по волоконно-оптическим линиям связи (ВОЛС), как по атмосферным каналам, так и в космическом пространстве с помощью спутников. Принцип работы данных систем следующий. На стороне передатчика (Алиса) формируются квантовые состояния. Для этого используется когерентное лазерное излучение, ослабленное до одиночных фотонов с помощью  аттенюатора. В подготовленные кванты света вносится изменение в поляризацию или фазовый свдиг фотона. Подготовленное таким образом состояние передается по каналу связи к приемнику (Боб). На приемной стороне происходит независимое от Алисы повторное измерение состояния фотона. В случае корреляции у Боба принятый одиночный фотон регистрируется детектором одиночных фотонов. Благодаря свойствам одиночного фотона в виде невозможности клонирования, невозможности измерения без разрушения и его неделимости возможно отследить воздействие злоумышленника, так как его действия будут приводить к появлению ошибок в полученной битовой последовательности. Так обеспечивается контроль несанкционированного допуска. 

Отдельным классом выделяются системы квантового распределения ключа на непрерывных переменных (КРКНП). В таких системах квантовое состояние, подготовленное и переданное Алисой, на приемной стороне взаимодействует с сильным лазерным излучением. И результат этого взаимодействия регистрируется балансным детектором. Основными отличаями данного детектора от детектора одиночных фотонов является использование двух классических фотоприемников, подключенных таким образом, что их фототоки взаимнов вычитаются, что позволяет уменьшить шум системы,  и отстуствие охлаждения до температур порядка -40 $^{\circ}$  градусов Цельсия. Все это позволяет упростить конечную систему. К преимуществам КРКНП можно отнести большую скорость выработки секретного ключа по сравнению с системами КРК на дискретных переменных, в которых применяются детекторы одиночных фотонов. 

Среди сложностей систем КРКНП выделяется способ передачи сильного лазерного излучения или локального осциллятора (ЛО) на приемную сторону и его разделения с квантовым сигналом. В первых системах КРКНП с Гауссовой модуляцией Локальный осциллятор и квантовые состояния  генерировались у передатчика, объединялись и передавались совместно в квантовый канал. На приемной стороне локальный осциллятор и квантовый сигнал разделяются, ЛО задерживается специальной линией задержки и снова соединяются на светоделителе для взаимодействия. Результатом этого взаимодействия является интерфереционная картина, распределение интенсивности которой зависит от закодированного Алисой состояния. Полученное поле регистрируется балансным детектором, на выходе такого формируется уровень напряжения, который в дальнейшем подвергается пост-обработке.  Передача локального осциллятора через канал ограничивает дальность работы системы такого типа и ограничивает скорость выработки ключа, так как для лучшей работы системы необходим ЛО как можно большей мощности. Второй проблемой является возможности злоумышленника манипулировать локальным осциллятором для создания каналов утечки информации. В качестве альтернативы предлагается использовать локальный осциллятор, сгенерированный на приемной стороне. Такое решение позволит увеличить дальность передачи ключа, скорость его выработки и закрыть уявзимость к атаке на ЛО.


Одним из перспективных подходов к реализации систем квантовой коммуникации на непрерывных переменных является система квантовой коммуникации на боковых частотах модулированного излучения. В основе данного метода лежит вынесение квантового канала на боковые частоты, которые появляются в результате модуляции оптического излучения переменным электрическим полем. Благодаря этому повышается устойчивость передавемого сигнала ко внешним воздействиями и обеспечивается высокая спектральная эффективность, а также обеспечивается показатели по отношению скорости выработки ключа к дальности между блоками приемника и передатчика, сравнимые  с другими системами квантовой коммуникации. Данный метод подходит и для реализации протоколов на непрерывных переменных с когерентными методами детектирования. В частности, в данной работе рассматривается гетеродинный метод, при котором квантовые состояния, подготовленные Алисой, передаются по волоконной линии связи к приемнику, в нем попадают на светоделитель с формулой 2х2 и коэффициентом деления 50:50 и смешиваются на нем с мощнным локальным осциллятором, который отстроен по частоте от передающего лазера на величину, которая превышает частоту смены состояний. Результат интерференции регистрируется балансным детектором. На выходе балансного детектора формируется сигнал на промежуточной частоте от всего спектра сигнала, переданного Алисой. Для извлечения инфорамции требуется провести фильтрацию с помощью фильтра низких частот и демодуляцию полученного сигнала для генерации сырого ключа. 

Одной из проблем при реализации гетеродинного метода детектирования для распределения ключа является необходимость компенсации фазовых шумов. Для этого применяют различные методы. Первым из таких методов является передача "пилотного" импульса, при детектировании которого измеряется фазовый шум, внесенный каналом. После этого измеренное значение учитывается в постобработке состояний. Второе - это реализация обратной связи в различных формах. В рамках данной работы предлагается использовать метод оптической обратной связи для системы квантового распределения ключа на боковых частотах на непрерывных переменных. Суть данного метода заключается в инжекции лазерного излучения от ведущего лазера, который является лазером передатчика, в лазер ведомый, который используется в качестве локального осциллятора в приемнике. Данный метод позволяет стабилизировать длину волны ЛО и уменьшить фазовые шумы из-за того, что оба источника являются генераторами когеретного излучения со случайной фазой.

Метод оптической инжекции требует дополнительного канала для передачи создания обратной связи. Такой канал усложняет систему и повышает требования к волоконно-оптической линии связи (ВОЛС), что особенно критично в городских линиях связи, где выделение дополнительного волокна или канала в сетях с мультиплексированием затруднительно. Решением данной проблемы может являтся система квантового распределения ключа на непрерывных переменных с применением гетеродинного детектирования с независимым ЛО. Суть данной системы заключается в том, что на приемнике и передатичке установленны лазеры со стабилизацией длины волны и со шириной спектральной линии менее 10 кГц. Такой подход позволяет не прибегать к постоянной подстройке длин волн лазеров и уменьшить фазовый шум, связанный с независимостью источников излучения.
Однако, фазовый шум при этом не исчезает, поэтому его все еще необходимо компенсировать. В случае реализации такого метода детектирования сигналов для протокола квантового распределения ключа на боковых частотах для этого можно использовать несущую частоту, измеряя ее фазу и внося корректировки в постобработке. 

Отличия реальных систем КРК от  моделей, используемых для теориетических доказательств, могут быть использованы злоумышленником для проведения различных типов атак на оборудование, входящее в состав системы. В работах ранее было показано, что источники лазерного излучения на основе полупроводниковых кристаллов могут быть уязвимы к "засеву" внешним излучением злоумышленника на длине волны близкой к той, что использует передатчик. В результате этой атаки изменяется форма излучаемого импульса и увеличивается выходная мощность, в отдельных случаях можно наблюдать и изменение длины волны. Эти эффекты приводят к увеличению среднего числа фотонов, излучаемых передатчиком, что открывает возможность для злоумышленника атаки с ращеплением числа фотонов. 

Однако в литературе не рассматривались атака "засевом" лазерным излучением на других длинах волн. Атака такого типа опаснее тем, что для защиты от нее используются пассивные волоконно-оптические элементы, вносящие дополнительное затухание, например изоляторы или DWDM фильтры. Но существуют работы, которые демонстрируют, что величина затухания в таких элементах может уменьшаться при существенном изменении падающей длины волны излучения. Например, изолятор с рабочей длиной волны 1550 нм вносит 50 дБ потерь при обратном прохождении, когда при облучении излучением на длине волны 1310 нм эта величина составляет 20 дБ. А в случае с DWDM фильтром, он практически не вносит затухание на длине волны 1310 нм. Таким образом, злоумышленнику гораздо проще осуществить атаку "засевом" лазерным излучением, так как на данной длине волны вносимое затухание меньше. 

Такой тип атаки носит название "атака оптической накачкой". Ее суть заключается в том, что злоумышленник зондирует лазер длиной волны, отличой от рабочей. При этом это излучение поглащается активной средной лазера передатчика так, что поглощенное излучение выступает в роли оптической накачки, которая работает как дополнение к электрической накачки полупроводникового лазера. В этом случае изменяется Ватт-Амперная характеристика лазера и его квантовая эффективность. Это приводит к тому, что изменяется энергия излученных импульсов увеличивается при неизменной величине тока накачки. В рамках данной работы впервые обозначен данный тип атаки, определена нижняя граница необходимой мощности излучения на длние волны 1310 нм для изменения характеристик изучаемого лазера и измерено влияние оптической накачки на характеристики лазера. 

Существует решение
    
Цель работы.\\
Задачи работы.\\
Научная новизна работы.\\
Теоретическая и практическая значимость работы.\\
Положения выносимые на защиту.\\
Апробация работы.\\
Достоверность научных достижений.\\
Внедрение результатов работы.\\
Публикации.\\
Структура и объем диссертации. \\

