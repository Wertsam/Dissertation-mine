\chapter*{Введение}                         % Заголовок
\addcontentsline{toc}{chapter}{Введение}
\noindent 
\section*{Актуальность темы}
Квантовое распределения ключа (КРК) - актуальная технология, появившаяся из теории квантовой информатики, позволяющая распределить симметричную битовую последовательность с помощью квантовых методов у двух и более пользователей для использования этой последовательности в качестве ключа для симметричного шифрования данных и одновременным обнаружением несанкционированного доступа со стороны нелегитимных пользователей. Использование квантовых состояний света при распределении ключа позволяет достичь уровня секретности, недоступного для классических протоколов шифрования. Такие квантовые состояния могут быть представлены в виде одиночных фотонов. Их квантовые свойства не позволяют злоумышленнику скопировать их состояния или считать их без изменения и без внесения ошибок. Такие квантовые состояния возможно передавать как по волоконно-оптическим линиям связи (ВОЛС), как по атмосферным каналам, так и в космическом пространстве с помощью спутников. Принцип работы данных систем следующий. На стороне передатчика (Алиса) формируются квантовые состояния. Для этого используется когерентное лазерное излучение, ослабленное до одиночных фотонов с помощью аттенюатора. В подготовленные кванты света вносится изменение в поляризацию или фазовый сдвиг фотона. Подготовленное таким образом состояние передается по каналу связи к приемнику (Боб). На приемной стороне происходит независимое от Алисы повторное измерение состояния фотона. В случае корреляции у Боба принятый одиночный фотон регистрируется детектором одиночных фотонов. Благодаря свойствам одиночного фотона в виде невозможности клонирования, невозможности измерения без разрушения и его неделимости возможно отследить воздействие злоумышленника, так как его действия будут приводить к появлению ошибок в полученной битовой последовательности. Так обеспечивается контроль несанкционированного допуска. 

Отдельным классом выделяются системы квантового распределения ключа на непрерывных переменных (КРК-НП). В таких системах квантовое состояние, подготовленное и переданное Алисой, на приемной стороне взаимодействует с сильным лазерным излучением. И результат этого взаимодействия регистрируется балансным детектором. Основными отличиями данного детектора от детектора одиночных фотонов является использование двух классических фотоприемников, подключенных таким образом, что их фототоки взаимно вычитаются, что позволяет уменьшить шум системы, и отсутствие охлаждения до температур порядка -40$^{\circ}$  градусов Цельсия. Все это позволяет упростить конечную систему. К преимуществам КРК-НП можно отнести большую скорость выработки секретного ключа по сравнению с системами КРК на дискретных переменных, в которых применяются детекторы одиночных фотонов. 

Среди сложностей систем КРК-НП выделяется способ передачи сильного лазерного излучения или локального осциллятора (ЛО) на приемную сторону и его разделения с квантовым сигналом. В первых системах КРК-НП с Гауссовой модуляцией Локальный осциллятор и квантовые состояния генерировались у передатчика, объединялись и передавались совместно в квантовый канал. На приемной стороне локальный осциллятор и квантовый сигнал разделяются, ЛО задерживается специальной линией задержки и снова соединяются на светоделителе для взаимодействия. Результатом этого взаимодействия является интерференционная картина, распределение интенсивности которой зависит от закодированного Алисой состояния. Полученное поле регистрируется балансным детектором, на выходе такого формируется уровень напряжения, который в дальнейшем подвергается пост-обработке.  Передача локального осциллятора через канал ограничивает дальность работы системы такого типа и ограничивает скорость выработки ключа, так как для лучшей работы системы необходим ЛО как можно большей мощности. Второй проблемой является возможности злоумышленника манипулировать локальным осциллятором для создания каналов утечки информации. В качестве альтернативы предлагается использовать локальный осциллятор, сгенерированный на приемной стороне. Такое решение позволит увеличить дальность передачи ключа, скорость его выработки и закрыть уязвимость к атаке на ЛО.


Одним из перспективных подходов к реализации систем квантовой коммуникации на непрерывных переменных является система квантовой коммуникации на боковых частотах модулированного излучения. В основе данного метода лежит вынесение квантового канала на боковые частоты, которые появляются в результате модуляции оптического излучения переменным электрическим полем. Благодаря этому повышается устойчивость передаваемого сигнала ко внешним воздействиям и обеспечивается высокая спектральная эффективность, а также обеспечивается показатели по отношению скорости выработки ключа к дальности между блоками приемника и передатчика, сравнимые с другими системами квантовой коммуникации. Данный метод подходит и для реализации протоколов на непрерывных переменных с когерентными методами детектирования. В частности, в данной работе рассматривается гетеродинный метод, при котором квантовые состояния, подготовленные Алисой, передаются по волоконной линии связи к приемнику, в нем попадают на светоделитель с формулой 2х2 и коэффициентом деления 50:50 и смешиваются на нем с мощным локальным осциллятором, который отстроен по частоте от передающего лазера на величину, которая превышает частоту смены состояний. Результат интерференции регистрируется балансным детектором. На выходе балансного детектора формируется сигнал на промежуточной частоте от всего спектра сигнала, переданного Алисой. Для извлечения информации требуется провести фильтрацию с помощью фильтра низких частот и демодуляцию полученного сигнала для генерации сырого ключа. 

Одной из проблем при реализации гетеродинного метода детектирования для распределения ключа является необходимость компенсации фазовых шумов. Для этого применяют различные методы. Первым из таких методов является передача "пилотного" импульса, при детектировании которого измеряется фазовый шум, внесенный каналом. После этого измеренное значение учитывается в постобработке состояний. Второе - это реализация обратной связи в различных формах. В рамках данной работы предлагается использовать метод оптической обратной связи для системы квантового распределения ключа на боковых частотах на непрерывных переменных. Суть данного метода заключается в инжекции лазерного излучения от ведущего лазера, который является лазером передатчика, в лазер ведомый, который используется в качестве локального осциллятора в приемнике. Данный метод позволяет стабилизировать длину волны ЛО и уменьшить фазовые шумы из-за того, что оба источника являются генераторами когерентного излучения со случайной фазой.

Метод оптической инжекции требует дополнительного канала для передачи создания обратной связи. Такой канал усложняет систему и повышает требования к волоконно-оптической линии связи (ВОЛС), что особенно критично в городских линиях связи, где выделение дополнительного волокна или канала в сетях с мультиплексированием затруднительно. Решением данной проблемы может является система квантового распределения ключа на непрерывных переменных с применением гетеродинного детектирования с независимым ЛО. Суть данной системы заключается в том, что на приемнике и передатчике установлены лазеры со стабилизацией длины волны и со шириной спектральной линии менее 10 кГц. Такой подход позволяет не прибегать к постоянной подстройке длин волн лазеров и уменьшить фазовый шум, связанный с независимостью источников излучения.
Однако, фазовый шум при этом не исчезает, поэтому его все еще необходимо компенсировать. В случае реализации такого метода детектирования сигналов для протокола квантового распределения ключа на боковых частотах для этого можно использовать несущую частоту, измеряя ее фазу и внося корректировки в постобработке. 

Отличия реальных систем КРК от моделей, используемых для теоретических доказательств, могут быть использованы злоумышленником для проведения различных типов атак на оборудование, входящее в состав системы. В работах ранее было показано, что источники лазерного излучения на основе полупроводниковых кристаллов могут быть уязвимы к "засеву" внешним излучением злоумышленника на длине волны близкой к той, что использует передатчик. В результате этой атаки изменяется форма излучаемого импульса и увеличивается выходная мощность, в отдельных случаях можно наблюдать и изменение длины волны. Эти эффекты приводят к увеличению среднего числа фотонов, излучаемых передатчиком, что открывает возможность для злоумышленника атаки с расщеплением числа фотонов. 

Однако в литературе не рассматривались атака "засевом" лазерным излучением на других длинах волн. Атака такого типа опаснее тем, что для защиты от нее используются пассивные волоконно-оптические элементы, вносящие дополнительное затухание, например, изоляторы или DWDM фильтры. Но существуют работы, которые демонстрируют, что величина затухания в таких элементах может уменьшаться при существенном изменении падающей длины волны излучения. Например, изолятор с рабочей длиной волны 1550 нм вносит 50 дБ потерь при обратном прохождении, когда при облучении излучением на длине волны 1310 нм эта величина составляет 20 дБ. А в случае с DWDM фильтром, он практически не вносит затухание на длине волны 1310 нм. Таким образом, злоумышленнику гораздо проще осуществить атаку "засевом" лазерным излучением, так как на данной длине волны вносимое затухание меньше. 

Такой тип атаки носит название "атака оптической накачкой". Ее суть заключается в том, что злоумышленник зондирует лазер длиной волны, отличной от рабочей. При этом это излучение поглощается активной средой лазера передатчика так, что поглощенное излучение выступает в роли оптической накачки, которая работает как дополнение к электрической накачки полупроводникового лазера. В этом случае изменяется Ватт-Амперная характеристика лазера и его квантовая эффективность. Это приводит к тому, что изменяется энергия излученных импульсов увеличивается при неизменной величине тока накачки. В рамках данной работы впервые обозначен данный тип атаки, определена нижняя граница необходимой мощности излучения на длине волны 1310 нм для изменения характеристик изучаемого лазера и измерено влияние оптической накачки на характеристики лазера. 

В системах квантового распределения применяются источники лазерного излучения на основе оптической инжекции. Такие источники построены следующим образом: применяются два лазера - ведущий и ведомый, соединенных циркуляторном. Излучение ведомого лазера позволяет снизить дрожание излучаемых импульсов, стабилизировать мощность выходного излучения и сузить спектральную линию. Однако такие источники не исследовались на устойчивость ко внешнему излучению. Ранее показанные работы по лазерному "засеву" были проведены только для одиночных источников излучения. Источник, построенный на основе оптической инжекции, имеет несколько преимуществ относительного одиночного: наличие изоляции от квантового канала за счет оптического циркулятора и наличие внешнего излучения ведущего лазера. В рамках данной работы изучается влияние мощного лазерного излучения на длительность, дрожание и амплитуду излучаемых импульсов, продемонстрирована нижняя граница мощности излучения необходимого  для внесения изменений в работу данной системы.
\section*{Цель работы}
Разработать систему гетеродинного приема сигналов в квантовой системе коммуникаций на боковых частотах с локальным осциллятором на стороне получателя с применением оптической инжекции и исследовать устойчивость к атакам на техническую реализацию источников лазерного излучения в этой системе. 

\section*{Задачи работы}
\textbf{Задача 1}\\
Реализация обратной связи в виде оптической инжекции для системы КРК на боковых частотах с гетеродинным методом детектирования и применением непрерывных переменных. 

\textbf{Задача 2}\\
Применение гетеродинного приема сигналов в системах КРК, и гетеродинное детектирование мультиплексированного сигнала на одной несущей\\

\textbf{Задача 3}\\
Исследовать атаку оптической накачкой на источники излучения, которые могут являться локальным осциллятором для систем квантового распределения ключа на непрерывных переменных\\

\textbf{Задача 4}\\
Исследовать влияние мощного оптического излучения на источник излучения на основе оптической инжекции\\

\section*{Научная новизна}
Впервые реализована система обратной связи с помощью оптической инжекции для системы квантового распределения ключей на боковых частотах и передан просеянный ключ. Реализован гетеродинный метод детектирования сигналов с двумя независимыми источниками излучения для системы квантового распределения ключей на боковых частотах и разработан алгоритм контроля поляризации для этой системы. Впервые продемонстрирован новый тип атаки на техническую реализацию - атака оптической накачкой на источник излучения в системах квантового распределения ключей, которая позволяет увеличить излучаемое среднее число фотонов в обход существующих методов защиты. Определено экспериментально влияние мощного лазерного излучения на источник когерентного излучения на основе оптической инжекции, увеличивающее энергию излучаемых импульсов и ее разброс, увеличивает выходную мощность атакуемого источника, что в совокупности приводит к снижению скорости выработки секретного ключа.
\section*{Теоретическая и практическая значимость}
Теоретическая значимость работы определяется тем, что в рамках  ее  были переданы фазово-кодированные состояния в системе квантового распределения ключей на боковых частотах на непрерывных переменных с гетеродинным методом детектирования сигналов и были стабилизированы длины волн информационного лазера и лазера локального осциллятора. Также в рамках работы был совершен обмен фазово-кодированными состояниями в системе квантового распределения ключей на боковых частотах на непрерывных переменных с гетеродинным методом детектирования сигналов и двумя независимыми источниками излучения информационного сигнала и локального осциллятора, в рамках передачи таких состояний отработан алгоритм подстройки поляризации информационного излучения. Увеличена выходная средняя мощность и энергия импульсов лазера с распределенной обратной связью, используемого в системах квантового распределения ключей, с помощью оптической накачки на длине волны 1310 нм. Увеличена средняя выходная мощность и среднеквадратическое отклонение амплитуды выходных импульсов, излучаемых источником когерентного излучения на основе оптической инжекции, с помощью мощного лазерного излучения злоумышленника, приводящее к созданию дополнительной уязвимости по доступу к секретному ключу. 
Практическая значимость работы заключается в том, что проведенные экспериментальные исследования по реализации гетеродинного метода детектирования сигналов показывают работоспособность данного подхода для создания систем квантового распределения ключей с применением такого способа регистрации сигналов. Исследованные же методы воздействия злоумышленника на источники излучения в системах квантового распределения ключей позволяет усовершенствовать модель нарушителя, повысив устойчивость конечных систем квантового распределения ключей к атакам на техническую реализацию. 

\section*{Основные положения, выносимые на защиту}
\begin{enumerate}
    \item Передача фазово-кодированных сигналов в системе квантового распределения ключей на непрерывных переменных с гетеродинным методом детектирования сигналов и локальным осциллятором, реализованным на стороне приемника, становится возможной при стабилизации длин волн используемых источников излучения за счет применения метода оптической инжекции для реализации обратной связи.
    \item Алгоритм, заключающийся в контроле поляризации входящего сигнала,  основанный на анализе спектрального состава электрического сигнала, полученного после Быстрого Преобразования Фурье, и с поворотом поляризации на основе проведенного анализа, позволяет произвести обмен фазово-кодированными состояниями в системе квантовой коммуникации на боковых частотах с применением непрерывных переменных и гетеродинным методом регистрации сигналов на основе двух независимых источников лазерного  излучения телекоммуникационного диапазона длин волн  и с применением частотного мультиплексирования на одной несущей частоте. 
    \item Поглощение излучения лазера нарушителя  активной средой полупроводникового лазера с распределенной обратной связью, используемого в передатчике системы квантового распределения ключей, приводит к увеличению излучаемого им среднего числа фотонов.
    \item Засеивание ведомого лазера в источнике излучения, построенного  на основе метода оптической инжекции, лазером нарушителя, который работает в непрерывном режиме, мощностью не менее 800 мВт и на длине волны, согласованной с длиной волны ведомого лазера,  повышает стандартное отклонение амплитуды выходных импульсов ведомого лазера на $3\%$, повышает стандартное отклонение их энергии на $3\%$, увеличивает стандартное отклонение длительности импульсов на $2.5\%$ и увеличивает среднюю излучаемую мощность на $8\%$, приводящее к снижению дальности передачи секретного ключа на $10\%$.
\end{enumerate}
%В работе проводились исследования по реализации новых подходов к регистрации сигналов для системы квантового распределения ключей на боковых частотах и исследовались атаки на техническую реализацию данной системы. В результате чего получены новые практические результаты, научная новизна которых заключается в том, что\\
%\textbf{Научная новизна 1}
%Разработан метод гетеродинного детектирования сигналов для системы квантового распределения ключей на боковых частотах с использованием непрерывных переменных на основе двух независимых источников излучения и алгоритм контроля поляризации этих источников лазерного излучения.\\
%\textbf{Научная новизна 2}
%Разработан метод гетеродинного детектирования сигналов с применением оптической инжекции для системы квантового распределения ключей на боковых частотах и сформирован протокол распределения секретных бит.\\
%\textbf{Научная новизна 3}
%Впервые проведено исследование оптической накачки злоумышленником лазера в системе квантового распределения ключей. \\
%\textbf{Научная новизна 4}
%Впервые изучено влияние мощного лазерного излучения на источник когерентного излучения на основе оптической инжекции в составе системы квантового распределения ключей.
\section*{Апробация результатов работы}
Основные результаты по теме диссертации докладывались на следующих конференциях:
\begin{enumerate}
    \item КМУ Х 'Применение гетеродинного метода анализа сигналов для реализации протокола квантовой коммуникации с топологией 'звезда'
    \item ФЭКС-2021 'Применение гетеродинного метода анализа сигналов для реализации протокола квантовой коммуникации с топологией 'звезда'
    \item ППС LI 'Когерентный прием в системах квантовой коммуникации на боковых частотах с недоверенным приёмным узлом'
    \item XI КМУ 'Многопользовательские квантовые сети городского масштаба на основе пассивных оптических сетей'
    \item 20th International Conference Laser Optics ICLO 2022 'Continuous variable measurement-device-independent quantum communication scheme based on subcarrier waves'
    \item XII КМУ 'Система квантовой коммуникации на непрерывных переменных с недоверенным приемным узлом'
    \item LII научная и учебно-методическая конференция ППС 'Частотное мультиплексирование для системы квантового распределения ключа на боковых частотах'
    \item Всероссийская научная конференция с международным участием 'Невская фотоника-2023' (09.10.2023 - 13.10.2023), 'Гетеродинное детектирование для системы квантового распределения ключа на боковых частотах с двумя независимыми источниками излучения'
    \item 22th International Conference Laser Optics ICLO 2024 'Laser-pumping attack on QKD sources', 1-5 июля 2024 г.
    \item 22th International Conference Laser Optics ICLO 2024, 'Secure laser source for QKD systems', 1-5 июля 2024 г.
    \item QCrypt 2024, 2-6.09.24, 'Optical pumping attack to laser source in Quantum key distribution system'
\end{enumerate}



\section*{Достоверность}
Достоверность полученных результатов основана на использовании современных методов научного исследования и сравнении полученных результатов с данными научно-технической литературы. При проведении исследований применялись утвержденные методики и аттестованное оборудование. Обработка экспериментальных данных осуществлялись при помощи пакета прикладного программного обеспечения Origin и Питон. Материалы опубликованы в 9 печатных работах, а также были представлены на 10 международных и российских конференциях.
\section*{Внедрение результатов работы}
Результаты диссертационной работы внедрены в проекты, выполняемые в рамках Дорожной карты по направлению "Квантовые коммуникации", таких как  "Разработка и создание системы квантовой коммуникации на непрерывных переменных", в котором внедрены результаты, полученные по реализации гетеродинного метода детектирования для системы квантового распределения ключей на боковых частотах  и "Создание Пилотного участка Магистральной Квантовой сети", в рамках которого внедрены исследования устойчивости источника лазерного излучения к атаке оптической накачкой


\section*{Личный вклад автора}
Аспирантом лично разработаны оптические схемы систем квантового распределения ключей на боковых частотах с гетеродинным методом детектирования сигналов и с применением метода оптической инжекции, а также исследовано влияние оптической накачки на лазер с распределенной обратной связью и изучено влияние мощного оптического излучения на источники излучения на основе оптической инжекции. Аспирантом были проведены экспериментальные работы по изучению работы разработанных систем и самостоятельно обработаны экспериментальные результаты.
\section*{Структура и объем диссертации} Диссертация состоит из введения, 4 глав, заключения и 2 приложений. Полный объем диссертации составляет 311 страниц, c  \totalfigures\ рисунком  и \totaltables\ таблицами. Список литературы содержит 120 наименований.
\section*{Публикации по теме работы}
Основные результаты по теме диссертации изложены в 9 публикациях. Из них 9 опубликованы в изданиях, индексируемых в базе цитирования Scopus. Также имеется 4 патента на изобретения.\\
