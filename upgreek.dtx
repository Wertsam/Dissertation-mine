%\CheckSum{281}
%
% \iffalse
%
% File `upgreek.dtx'.
% Copyright (c) 2001--2003 Walter Schmidt
%
% This program may be distributed and/or modified under the
% conditions of the LaTeX Project Public License, either version 1.2
% of this license or (at your option) any later version.
% The latest version of this license is in
%   http://www.latex-project.org/lppl.txt
% and version 1.2 or later is part of all distributions of LaTeX
% version 1999/12/01 or later.
%
% This program consists of the files upgreek.dtx and upgreek.ins
%
% \fi
%
% \iffalse
%
%<*driver>
\ProvidesFile{upgreek.drv}
%</driver>
%<package>\ProvidesPackage{upgreek}
  [2003/02/12 v2.0 (WaS)]
%<*driver>
\documentclass{ltxdoc}
\CodelineNumbered
\begin{document}
 \DocInput{upgreek.dtx}
\end{document}
%</driver>
% \fi
%
% \GetFileInfo{upgreek.drv}
% \DeleteShortVerb{\|}
% \MakeShortVerb{\+}
%
% \title{The \textsf{upgreek} package for \LaTeXe}
% \author{Walter Schmidt\thanks{{\ttfamily w.a.schmidt@gmx.net}}}
% \date{(\fileversion{} -- \filedate)}
% \maketitle
% \thispagestyle{empty}
% \sloppy
%
% \noindent
% The default CM math fonts used by \TeX\ do not include upright
% lowercase Greek characters, and many alternative math font sets don't,
% either.  (There are math fonts providing \emph{only} an upright
% Greek alphabet, but that's a different story.) 
% However, mathematical constants such as $\pi = 3.14\dots$ 
% are sometimes required to be typeset in roman (i.e., upright) style,
% or one may need upright Greek to designate elementary particles.
%
% As a workaround, 
% the \LaTeX{} package \textsf{upgreek} makes the upright
% Greek characters from the `Euler' or `Adobe Symbol' typefaces available
% as math symbols.  The lowercase letters are named  
% +\upalpha+, +\upbeta+ etc., whereas +\Updelta+, +\Upgamma+ etc.\ create
% upper case. 
%
% Just like +\alpha+, +\beta+ etc., these symbols work only in math
% mode, and their size is properly adjusted, when they are used in 
% superscripts, subscripts or fractions.
%
% At first sight, providing also the uppercase letters seems pointless,
% since +\Delta+, +\Gamma+, etc.\ are also upright by default.  However,
% it may be useful to have all upright Greek letters available in the
% same style.  Note that the uppercase letters were intentionally
% \emph{not} named +\upDelta+ etc., so as not to clash with other
% existent macro packages.
%
% The typeface is selected using a package option, so you can
% choose what blends best with the other fonts used in your document:
% \begin{description}
%   \item[{\mdseries\ttfamily [Euler]}] Euler Roman/Bold (default)
%   \item[{\mdseries\ttfamily [Symbol]}] Adobe Symbol
%   \item[{\mdseries\ttfamily [Symbolsmallscale]}] Adobe Symbol, scaled down to 90\% of
%   its natural size. (New feature in version 2.0\,!)
% \end{description}
% 
% Note that +\upmu+ should normally \emph{not} be used as the prefix
% for physical units, meaning $10^{-6}$.  The mu symbol to be used there
% is to be taken from the \emph{text} font, and
% most Latin text fonts do actually include a suitable Greek mu,
% which can be accessed as +\textmu+.
% Most likely -- depending on the encoding of your text fonts -- you
% need to load the package \textsf{textcomp} for this purpose.
% If, however, there is no mu in your text fonts,
% using +\upmu+ is still better than nothing.
% See also the \textsf{gensymb} package, which is distributed from the same
% CTAN directory than \textsf{upgreek}:  It provides a command for `micro' 
% that works in both text and math mode and uses either +\textmu+ or +\upmu+, 
% depending on the available character sets.
% 
% 
% \StopEventually{}
% 
% \section*{The package code}
% 
%    \begin{macrocode}
%<*package>
\DeclareOption{Symbol}{\let\uppi=s}
\DeclareOption{Symbolsmallscale}{\let\uppi m}
\DeclareOption{Euler}{\let\uppi=e}
\ExecuteOptions{Euler}
\ProcessOptions\relax
\ifx\uppi e
  \PackageInfo{upgreek}{Using Euler Roman for upright Greek}
  \DeclareFontFamily{U}{eur}{\skewchar\font'177}
  \DeclareFontShape{U}{eur}{m}{n}{%
    <-6> eurm5 <6-8> eurm7 <8-> eurm10}{}
  \DeclareFontShape{U}{eur}{b}{n}{%
    <-6> eurb5 <6-8> eurb7 <8-> eurb10}{}
  \DeclareSymbolFont{ugrf@m}{U}{eur}{m}{n}
  \SetSymbolFont{ugrf@m}{bold}{U}{eur}{b}{n}
  \let\uppi\@undefined
  \DeclareMathSymbol{\upalpha}{\mathord}{ugrf@m}{"0B}
  \DeclareMathSymbol{\upbeta}{\mathord}{ugrf@m}{"0C}
  \DeclareMathSymbol{\upgamma}{\mathord}{ugrf@m}{"0D}
  \DeclareMathSymbol{\updelta}{\mathord}{ugrf@m}{"0E}
  \DeclareMathSymbol{\upepsilon}{\mathord}{ugrf@m}{"0F}
  \DeclareMathSymbol{\upzeta}{\mathord}{ugrf@m}{"10}
  \DeclareMathSymbol{\upeta}{\mathord}{ugrf@m}{"11}
  \DeclareMathSymbol{\uptheta}{\mathord}{ugrf@m}{"12}
  \DeclareMathSymbol{\upiota}{\mathord}{ugrf@m}{"13}
  \DeclareMathSymbol{\upkappa}{\mathord}{ugrf@m}{"14}
  \DeclareMathSymbol{\uplambda}{\mathord}{ugrf@m}{"15}
  \DeclareMathSymbol{\upmu}{\mathord}{ugrf@m}{"16}
  \DeclareMathSymbol{\upnu}{\mathord}{ugrf@m}{"17}
  \DeclareMathSymbol{\upxi}{\mathord}{ugrf@m}{"18}
  \DeclareMathSymbol{\uppi}{\mathord}{ugrf@m}{"19}
  \DeclareMathSymbol{\uprho}{\mathord}{ugrf@m}{"1A}
  \DeclareMathSymbol{\upsigma}{\mathord}{ugrf@m}{"1B}
  \DeclareMathSymbol{\uptau}{\mathord}{ugrf@m}{"1C}
  \DeclareMathSymbol{\upupsilon}{\mathord}{ugrf@m}{"1D}
  \DeclareMathSymbol{\upphi}{\mathord}{ugrf@m}{"1E}
  \DeclareMathSymbol{\upchi}{\mathord}{ugrf@m}{"1F}
  \DeclareMathSymbol{\uppsi}{\mathord}{ugrf@m}{"20}
  \DeclareMathSymbol{\upomega}{\mathord}{ugrf@m}{"21}
  \DeclareMathSymbol{\upvarepsilon}{\mathord}{ugrf@m}{"22}
  \DeclareMathSymbol{\upvartheta}{\mathord}{ugrf@m}{"23}
  \DeclareMathSymbol{\upvarpi}{\mathord}{ugrf@m}{"24}
  \let\upvarrho\uprho
  \let\upvarsigma\upsigma
  \DeclareMathSymbol{\upvarphi}{\mathord}{ugrf@m}{"27}
  \DeclareMathSymbol{\Upgamma}{\mathord}{ugrf@m}{"00}
  \DeclareMathSymbol{\Updelta}{\mathord}{ugrf@m}{"01}
  \DeclareMathSymbol{\Uptheta}{\mathord}{ugrf@m}{"02}
  \DeclareMathSymbol{\Uplambda}{\mathord}{ugrf@m}{"03}
  \DeclareMathSymbol{\Upxi}{\mathord}{ugrf@m}{"04}
  \DeclareMathSymbol{\Uppi}{\mathord}{ugrf@m}{"05}
  \DeclareMathSymbol{\Upsigma}{\mathord}{ugrf@m}{"06}
  \DeclareMathSymbol{\Upupsilon}{\mathord}{ugrf@m}{"07}
  \DeclareMathSymbol{\Upphi}{\mathord}{ugrf@m}{"08}
  \DeclareMathSymbol{\Uppsi}{\mathord}{ugrf@m}{"09}
  \DeclareMathSymbol{\Upomega}{\mathord}{ugrf@m}{"0A}
\else  
  \ifx\uppi s
    \PackageInfo{upgreek}{Using Adobe Symbol for upright Greek}
    \DeclareSymbolFont{ugrf@m}{U}{psy}{m}{n}
  \else % m
    \PackageInfo{upgreek}{Using Adobe Symbol, scaled 900, for upright Greek}
    \DeclareFontFamily{U}{fsy}{}
      \DeclareFontShape{U}{fsy}{m}{n}{<->s*[.9]psyr}{}
    \DeclareSymbolFont{ugrf@m}{U}{fsy}{m}{n}
  \fi
  \let\uppi\@undefined
  \DeclareMathSymbol{\upalpha}{\mathord}{ugrf@m}{`a}
  \DeclareMathSymbol{\upbeta}{\mathord}{ugrf@m}{`b}
  \DeclareMathSymbol{\upgamma}{\mathord}{ugrf@m}{`g}
  \DeclareMathSymbol{\updelta}{\mathord}{ugrf@m}{`d}
  \DeclareMathSymbol{\upepsilon}{\mathord}{ugrf@m}{`e}
  \DeclareMathSymbol{\upzeta}{\mathord}{ugrf@m}{`z}
  \DeclareMathSymbol{\upeta}{\mathord}{ugrf@m}{`h}
  \DeclareMathSymbol{\uptheta}{\mathord}{ugrf@m}{`q}
  \DeclareMathSymbol{\upiota}{\mathord}{ugrf@m}{`i}
  \DeclareMathSymbol{\upkappa}{\mathord}{ugrf@m}{`k}
  \DeclareMathSymbol{\uplambda}{\mathord}{ugrf@m}{`l}
  \DeclareMathSymbol{\upmu}{\mathord}{ugrf@m}{`m}
  \DeclareMathSymbol{\upnu}{\mathord}{ugrf@m}{`n}
  \DeclareMathSymbol{\upxi}{\mathord}{ugrf@m}{`x}
  \DeclareMathSymbol{\uppi}{\mathord}{ugrf@m}{`p}
  \DeclareMathSymbol{\uprho}{\mathord}{ugrf@m}{`r}
  \DeclareMathSymbol{\upsigma}{\mathord}{ugrf@m}{`s}
  \DeclareMathSymbol{\uptau}{\mathord}{ugrf@m}{`t}
  \DeclareMathSymbol{\upupsilon}{\mathord}{ugrf@m}{`u}
  \DeclareMathSymbol{\upphi}{\mathord}{ugrf@m}{`f}
  \DeclareMathSymbol{\upchi}{\mathord}{ugrf@m}{`c}
  \DeclareMathSymbol{\uppsi}{\mathord}{ugrf@m}{`y}
  \DeclareMathSymbol{\upomega}{\mathord}{ugrf@m}{`w}
  \let\upvarepsilon\upepsilon
  \DeclareMathSymbol{\upvartheta}{\mathord}{ugrf@m}{`J}
  \DeclareMathSymbol{\upvarpi}{\mathord}{ugrf@m}{`v}
  \let\upvarrho\uprho
  \let\upvarsigma\upsigma
  \DeclareMathSymbol{\upvarphi}{\mathord}{ugrf@m}{`j}
  \DeclareMathSymbol{\Upgamma}{\mathord}{ugrf@m}{`G}
  \DeclareMathSymbol{\Updelta}{\mathord}{ugrf@m}{`D}
  \DeclareMathSymbol{\Uptheta}{\mathord}{ugrf@m}{`Q}
  \DeclareMathSymbol{\Uplambda}{\mathord}{ugrf@m}{`L}
  \DeclareMathSymbol{\Upxi}{\mathord}{ugrf@m}{`X}
  \DeclareMathSymbol{\Uppi}{\mathord}{ugrf@m}{`P}
  \DeclareMathSymbol{\Upsigma}{\mathord}{ugrf@m}{`S}
  \DeclareMathSymbol{\Upupsilon}{\mathord}{ugrf@m}{`U}
  \DeclareMathSymbol{\Upphi}{\mathord}{ugrf@m}{`F}
  \DeclareMathSymbol{\Uppsi}{\mathord}{ugrf@m}{`Y}
  \DeclareMathSymbol{\Upomega}{\mathord}{ugrf@m}{`W}
\fi
%</package>
%    \end{macrocode}
%
% The next line of code is only to prevent DocStrip from adding the
% character table to all modules:
%    \begin{macrocode}
\endinput
%    \end{macrocode}
% \Finale
%% \CharacterTable
%%  {Upper-case    \A\B\C\D\E\F\G\H\I\J\K\L\M\N\O\P\Q\R\S\T\U\V\W\X\Y\Z
%%   Lower-case    \a\b\c\d\e\f\g\h\i\j\k\l\m\n\o\p\q\r\s\t\u\v\w\x\y\z
%%   Digits        \0\1\2\3\4\5\6\7\8\9
%%   Exclamation   \!     Double quote  \"     Hash (number) \#
%%   Dollar        \$     Percent       \%     Ampersand     \&
%%   Acute accent  \'     Left paren    \(     Right paren   \)
%%   Asterisk      \*     Plus          \+     Comma         \,
%%   Minus         \-     Point         \.     Solidus       \/
%%   Colon         \:     Semicolon     \;     Less than     \<
%%   Equals        \=     Greater than  \>     Question mark \?
%%   Commercial at \@     Left bracket  \[     Backslash     \\
%%   Right bracket \]     Circumflex    \^     Underscore    \_
%%   Grave accent  \`     Left brace    \{     Vertical bar  \|
%%   Right brace   \}     Tilde         \~}
%%
